\documentclass[]{article}
\usepackage{lmodern}
\usepackage{amssymb,amsmath}
\usepackage{ifxetex,ifluatex}
\usepackage{fixltx2e} % provides \textsubscript
\ifnum 0\ifxetex 1\fi\ifluatex 1\fi=0 % if pdftex
  \usepackage[T1]{fontenc}
  \usepackage[utf8]{inputenc}
\else % if luatex or xelatex
  \ifxetex
    \usepackage{mathspec}
  \else
    \usepackage{fontspec}
  \fi
  \defaultfontfeatures{Ligatures=TeX,Scale=MatchLowercase}
\fi
% use upquote if available, for straight quotes in verbatim environments
\IfFileExists{upquote.sty}{\usepackage{upquote}}{}
% use microtype if available
\IfFileExists{microtype.sty}{%
\usepackage{microtype}
\UseMicrotypeSet[protrusion]{basicmath} % disable protrusion for tt fonts
}{}
\usepackage[margin=1in]{geometry}
\usepackage{hyperref}
\hypersetup{unicode=true,
            pdftitle={HW2},
            pdfauthor={Jie Ren},
            pdfborder={0 0 0},
            breaklinks=true}
\urlstyle{same}  % don't use monospace font for urls
\usepackage{color}
\usepackage{fancyvrb}
\newcommand{\VerbBar}{|}
\newcommand{\VERB}{\Verb[commandchars=\\\{\}]}
\DefineVerbatimEnvironment{Highlighting}{Verbatim}{commandchars=\\\{\}}
% Add ',fontsize=\small' for more characters per line
\usepackage{framed}
\definecolor{shadecolor}{RGB}{248,248,248}
\newenvironment{Shaded}{\begin{snugshade}}{\end{snugshade}}
\newcommand{\KeywordTok}[1]{\textcolor[rgb]{0.13,0.29,0.53}{\textbf{#1}}}
\newcommand{\DataTypeTok}[1]{\textcolor[rgb]{0.13,0.29,0.53}{#1}}
\newcommand{\DecValTok}[1]{\textcolor[rgb]{0.00,0.00,0.81}{#1}}
\newcommand{\BaseNTok}[1]{\textcolor[rgb]{0.00,0.00,0.81}{#1}}
\newcommand{\FloatTok}[1]{\textcolor[rgb]{0.00,0.00,0.81}{#1}}
\newcommand{\ConstantTok}[1]{\textcolor[rgb]{0.00,0.00,0.00}{#1}}
\newcommand{\CharTok}[1]{\textcolor[rgb]{0.31,0.60,0.02}{#1}}
\newcommand{\SpecialCharTok}[1]{\textcolor[rgb]{0.00,0.00,0.00}{#1}}
\newcommand{\StringTok}[1]{\textcolor[rgb]{0.31,0.60,0.02}{#1}}
\newcommand{\VerbatimStringTok}[1]{\textcolor[rgb]{0.31,0.60,0.02}{#1}}
\newcommand{\SpecialStringTok}[1]{\textcolor[rgb]{0.31,0.60,0.02}{#1}}
\newcommand{\ImportTok}[1]{#1}
\newcommand{\CommentTok}[1]{\textcolor[rgb]{0.56,0.35,0.01}{\textit{#1}}}
\newcommand{\DocumentationTok}[1]{\textcolor[rgb]{0.56,0.35,0.01}{\textbf{\textit{#1}}}}
\newcommand{\AnnotationTok}[1]{\textcolor[rgb]{0.56,0.35,0.01}{\textbf{\textit{#1}}}}
\newcommand{\CommentVarTok}[1]{\textcolor[rgb]{0.56,0.35,0.01}{\textbf{\textit{#1}}}}
\newcommand{\OtherTok}[1]{\textcolor[rgb]{0.56,0.35,0.01}{#1}}
\newcommand{\FunctionTok}[1]{\textcolor[rgb]{0.00,0.00,0.00}{#1}}
\newcommand{\VariableTok}[1]{\textcolor[rgb]{0.00,0.00,0.00}{#1}}
\newcommand{\ControlFlowTok}[1]{\textcolor[rgb]{0.13,0.29,0.53}{\textbf{#1}}}
\newcommand{\OperatorTok}[1]{\textcolor[rgb]{0.81,0.36,0.00}{\textbf{#1}}}
\newcommand{\BuiltInTok}[1]{#1}
\newcommand{\ExtensionTok}[1]{#1}
\newcommand{\PreprocessorTok}[1]{\textcolor[rgb]{0.56,0.35,0.01}{\textit{#1}}}
\newcommand{\AttributeTok}[1]{\textcolor[rgb]{0.77,0.63,0.00}{#1}}
\newcommand{\RegionMarkerTok}[1]{#1}
\newcommand{\InformationTok}[1]{\textcolor[rgb]{0.56,0.35,0.01}{\textbf{\textit{#1}}}}
\newcommand{\WarningTok}[1]{\textcolor[rgb]{0.56,0.35,0.01}{\textbf{\textit{#1}}}}
\newcommand{\AlertTok}[1]{\textcolor[rgb]{0.94,0.16,0.16}{#1}}
\newcommand{\ErrorTok}[1]{\textcolor[rgb]{0.64,0.00,0.00}{\textbf{#1}}}
\newcommand{\NormalTok}[1]{#1}
\usepackage{graphicx,grffile}
\makeatletter
\def\maxwidth{\ifdim\Gin@nat@width>\linewidth\linewidth\else\Gin@nat@width\fi}
\def\maxheight{\ifdim\Gin@nat@height>\textheight\textheight\else\Gin@nat@height\fi}
\makeatother
% Scale images if necessary, so that they will not overflow the page
% margins by default, and it is still possible to overwrite the defaults
% using explicit options in \includegraphics[width, height, ...]{}
\setkeys{Gin}{width=\maxwidth,height=\maxheight,keepaspectratio}
\IfFileExists{parskip.sty}{%
\usepackage{parskip}
}{% else
\setlength{\parindent}{0pt}
\setlength{\parskip}{6pt plus 2pt minus 1pt}
}
\setlength{\emergencystretch}{3em}  % prevent overfull lines
\providecommand{\tightlist}{%
  \setlength{\itemsep}{0pt}\setlength{\parskip}{0pt}}
\setcounter{secnumdepth}{0}
% Redefines (sub)paragraphs to behave more like sections
\ifx\paragraph\undefined\else
\let\oldparagraph\paragraph
\renewcommand{\paragraph}[1]{\oldparagraph{#1}\mbox{}}
\fi
\ifx\subparagraph\undefined\else
\let\oldsubparagraph\subparagraph
\renewcommand{\subparagraph}[1]{\oldsubparagraph{#1}\mbox{}}
\fi

%%% Use protect on footnotes to avoid problems with footnotes in titles
\let\rmarkdownfootnote\footnote%
\def\footnote{\protect\rmarkdownfootnote}

%%% Change title format to be more compact
\usepackage{titling}

% Create subtitle command for use in maketitle
\newcommand{\subtitle}[1]{
  \posttitle{
    \begin{center}\large#1\end{center}
    }
}

\setlength{\droptitle}{-2em}

  \title{HW2}
    \pretitle{\vspace{\droptitle}\centering\huge}
  \posttitle{\par}
    \author{Jie Ren}
    \preauthor{\centering\large\emph}
  \postauthor{\par}
      \predate{\centering\large\emph}
  \postdate{\par}
    \date{February 8, 2019}


\begin{document}
\maketitle

\begin{Shaded}
\begin{Highlighting}[]
\KeywordTok{rm}\NormalTok{(}\DataTypeTok{list=}\KeywordTok{ls}\NormalTok{())}
\KeywordTok{setwd}\NormalTok{(}\StringTok{"C:/Users/jiere/Dropbox/Spring 2019/ECON 613/ECON613_HW"}\NormalTok{)}
\KeywordTok{set.seed}\NormalTok{(}\DecValTok{100}\NormalTok{)}
\end{Highlighting}
\end{Shaded}

\subsection{Exercise 1: Data generation
---}\label{exercise-1-data-generation}

\begin{Shaded}
\begin{Highlighting}[]
\NormalTok{obs <-}\StringTok{ }\DecValTok{10000}
\NormalTok{X1 <-}\StringTok{ }\KeywordTok{runif}\NormalTok{(obs, }\DataTypeTok{max =} \DecValTok{3}\NormalTok{, }\DataTypeTok{min =} \DecValTok{1}\NormalTok{)}
\NormalTok{X2 <-}\StringTok{ }\KeywordTok{rgamma}\NormalTok{(obs,}\DecValTok{3}\NormalTok{,}\DataTypeTok{scale =} \DecValTok{2}\NormalTok{)}
\NormalTok{X3 <-}\StringTok{ }\KeywordTok{rbinom}\NormalTok{(obs,}\DecValTok{1}\NormalTok{,}\FloatTok{0.3}\NormalTok{)}
\NormalTok{eps <-}\StringTok{ }\KeywordTok{rnorm}\NormalTok{(obs, }\DataTypeTok{mean =} \DecValTok{2}\NormalTok{, }\DataTypeTok{sd =} \DecValTok{1}\NormalTok{)}
\NormalTok{Y <-}\StringTok{ }\FloatTok{0.5} \OperatorTok{+}\StringTok{ }\FloatTok{1.2}\OperatorTok{*}\NormalTok{X1 }\OperatorTok{-}\StringTok{ }\FloatTok{0.9}\OperatorTok{*}\NormalTok{X2 }\OperatorTok{+}\StringTok{ }\FloatTok{0.1}\OperatorTok{*}\NormalTok{X3 }\OperatorTok{+}\StringTok{ }\NormalTok{eps}
\NormalTok{ydum <-}\StringTok{ }\KeywordTok{ifelse}\NormalTok{(Y }\OperatorTok{>}\StringTok{ }\KeywordTok{mean}\NormalTok{(Y),}\DecValTok{1}\NormalTok{,}\DecValTok{0}\NormalTok{)}
\NormalTok{mydata <-}\StringTok{ }\KeywordTok{cbind}\NormalTok{(Y,ydum,X1,X2,X3,eps)}
\end{Highlighting}
\end{Shaded}

\subsection{Exercise 2: ---}\label{exercise-2}

\subsubsection{Correlation between Y and
X1}\label{correlation-between-y-and-x1}

\begin{Shaded}
\begin{Highlighting}[]
\KeywordTok{cor}\NormalTok{(Y,X1)}
\end{Highlighting}
\end{Shaded}

\begin{verbatim}
## [1] 0.2162074
\end{verbatim}

The correlation is limited from -1 to 1, so only thing we can tell is
that there is a strong positive correlation between these two variables,
which match the fact that the coef on X1 is possitive.

\subsubsection{Regression of Y on X where X = (1,X1,X2,X3)
(Mannually)}\label{regression-of-y-on-x-where-x-1x1x2x3-mannually}

\begin{Shaded}
\begin{Highlighting}[]
\NormalTok{ols <-}\StringTok{ }\ControlFlowTok{function}\NormalTok{(X,y,}\DataTypeTok{se=}\NormalTok{F)\{}
\NormalTok{  n <-}\StringTok{ }\KeywordTok{nrow}\NormalTok{(X)}
\NormalTok{  k <-}\StringTok{ }\KeywordTok{ncol}\NormalTok{(X)}
\NormalTok{  ols.coef <-}\StringTok{ }\KeywordTok{solve}\NormalTok{(}\KeywordTok{t}\NormalTok{(X)}\OperatorTok\NormalTok{X)}\OperatorTok\NormalTok{(}\KeywordTok{t}\NormalTok{(X)}\OperatorTok\NormalTok{y) }\CommentTok{#coefficient}
\NormalTok{  ols.res <-}\StringTok{ }\NormalTok{(y}\OperatorTok{-}\NormalTok{X}\OperatorTok\NormalTok{ols.coef) }\CommentTok{# residual}
\NormalTok{  ols.V <-}\StringTok{ }\DecValTok{1}\OperatorTok{/}\NormalTok{(n}\OperatorTok{-}\NormalTok{k) }\OperatorTok{*}\StringTok{ }\KeywordTok{as.numeric}\NormalTok{(}\KeywordTok{t}\NormalTok{(ols.res)}\OperatorTok\NormalTok{ols.res)}\OperatorTok{*}\KeywordTok{solve}\NormalTok{(}\KeywordTok{t}\NormalTok{(X)}\OperatorTok\NormalTok{X) }\CommentTok{#covariance matrix}
\NormalTok{  ols.se <-}\StringTok{ }\KeywordTok{as.matrix}\NormalTok{(}\KeywordTok{sqrt}\NormalTok{(}\KeywordTok{diag}\NormalTok{(ols.V))) }\CommentTok{#standard error}
  \KeywordTok{ifelse}\NormalTok{(se }\OperatorTok{==}\StringTok{ }\NormalTok{T, }\KeywordTok{return}\NormalTok{(}\KeywordTok{data.frame}\NormalTok{(}\DataTypeTok{b_hat =}\NormalTok{ ols.coef,}\DataTypeTok{se =}\NormalTok{ ols.se)),}\KeywordTok{return}\NormalTok{(}\KeywordTok{data.frame}\NormalTok{(}\DataTypeTok{b_hat =}\NormalTok{ ols.coef))) }\CommentTok{# output coef only by default}
\NormalTok{\}}

\NormalTok{X <-}\StringTok{ }\KeywordTok{cbind}\NormalTok{(}\DecValTok{1}\NormalTok{,X1,X2,X3)}
\NormalTok{y <-}\StringTok{ }\KeywordTok{as.matrix}\NormalTok{(Y)}
\NormalTok{ols.result <-}\StringTok{ }\KeywordTok{ols}\NormalTok{(X,y,}\DataTypeTok{se=}\NormalTok{T)}
\NormalTok{ols.result}
\end{Highlighting}
\end{Shaded}

\begin{verbatim}
##         b_hat          se
##     2.4561034 0.040982313
## X1  1.2158000 0.017491090
## X2 -0.8984434 0.002952839
## X3  0.1018762 0.022040052
\end{verbatim}

\subsubsection{Check with lm function}\label{check-with-lm-function}

\begin{Shaded}
\begin{Highlighting}[]
\NormalTok{lm.result <-}\StringTok{ }\KeywordTok{lm}\NormalTok{(Y}\OperatorTok{~}\NormalTok{X1}\OperatorTok{+}\NormalTok{X2}\OperatorTok{+}\NormalTok{X3, }\KeywordTok{data.frame}\NormalTok{(mydata))}
\KeywordTok{coef}\NormalTok{(}\KeywordTok{summary}\NormalTok{(lm.result))[,}\KeywordTok{c}\NormalTok{(}\DecValTok{1}\NormalTok{,}\DecValTok{2}\NormalTok{)] }\CommentTok{#coefficient & standard error}
\end{Highlighting}
\end{Shaded}

\begin{verbatim}
##               Estimate  Std. Error
## (Intercept)  2.4561034 0.040982313
## X1           1.2158000 0.017491090
## X2          -0.8984434 0.002952839
## X3           0.1018762 0.022040052
\end{verbatim}

\subsubsection{Boodstrap mannually}\label{boodstrap-mannually}

\begin{Shaded}
\begin{Highlighting}[]
\NormalTok{bootstrapse <-}\StringTok{ }\ControlFlowTok{function}\NormalTok{(n)\{}
  \CommentTok{# Input times of bootstrap}
\NormalTok{  boot.coef <-}\StringTok{ }\KeywordTok{data.frame}\NormalTok{(}\KeywordTok{matrix}\NormalTok{(}\DataTypeTok{nrow=}\DecValTok{4}\NormalTok{, }\DataTypeTok{ncol=}\DecValTok{1}\NormalTok{))[}\OperatorTok{-}\DecValTok{1}\NormalTok{]}
  \ControlFlowTok{for}\NormalTok{ (i }\ControlFlowTok{in} \DecValTok{1}\OperatorTok{:}\NormalTok{n) \{}
    \CommentTok{# sample from existing data to get X.s, y.s}
\NormalTok{    df.s <-}\StringTok{ }\NormalTok{mydata[}\KeywordTok{sample}\NormalTok{(}\KeywordTok{nrow}\NormalTok{(mydata),}\DataTypeTok{size =} \KeywordTok{nrow}\NormalTok{(mydata),}\DataTypeTok{replace =}\NormalTok{ T),]}
\NormalTok{    y.s <-}\StringTok{ }\NormalTok{df.s[,}\DecValTok{1}\NormalTok{]}
\NormalTok{    X.s <-}\StringTok{ }\KeywordTok{cbind}\NormalTok{(}\DecValTok{1}\NormalTok{,df.s[,}\KeywordTok{c}\NormalTok{(}\DecValTok{3}\OperatorTok{:}\DecValTok{5}\NormalTok{)])}
    \CommentTok{# calculate ols coefficient in each sample}
\NormalTok{    boot.coef <-}\StringTok{ }\KeywordTok{cbind}\NormalTok{(boot.coef,}\KeywordTok{ols}\NormalTok{(X.s,y.s))}
\NormalTok{  \}}
  \CommentTok{# Report the SE over all the obtained coefficients}
  \KeywordTok{return}\NormalTok{(}\KeywordTok{data.frame}\NormalTok{(}\DataTypeTok{se =} \KeywordTok{apply}\NormalTok{(boot.coef,}\DecValTok{1}\NormalTok{,sd)))}
\NormalTok{\}}

\NormalTok{boot.se_}\DecValTok{49}\NormalTok{ <-}\StringTok{ }\KeywordTok{bootstrapse}\NormalTok{(}\DecValTok{49}\NormalTok{)}
\NormalTok{boot.se_}\DecValTok{499}\NormalTok{ <-}\StringTok{ }\KeywordTok{bootstrapse}\NormalTok{(}\DecValTok{499}\NormalTok{)}
\NormalTok{boot.se_}\DecValTok{49}
\end{Highlighting}
\end{Shaded}

\begin{verbatim}
##             se
##    0.038829998
## X1 0.018566206
## X2 0.003008488
## X3 0.021760076
\end{verbatim}

\begin{Shaded}
\begin{Highlighting}[]
\NormalTok{boot.se_}\DecValTok{499}
\end{Highlighting}
\end{Shaded}

\begin{verbatim}
##            se
##    0.03734032
## X1 0.01666233
## X2 0.00292241
## X3 0.02206388
\end{verbatim}

\subsection{Exercise 3: ---}\label{exercise-3}

\subsubsection{likelihood function}\label{likelihood-function}

\begin{Shaded}
\begin{Highlighting}[]
\NormalTok{probit.llike <-}\StringTok{ }\ControlFlowTok{function}\NormalTok{(}\DataTypeTok{b. =}\NormalTok{ b, }\DataTypeTok{y. =}\NormalTok{ ydum,}\DataTypeTok{X. =}\NormalTok{ X)\{}
\NormalTok{  phi <-}\StringTok{ }\KeywordTok{pnorm}\NormalTok{(X.}\OperatorTok\NormalTok{b.)}
\NormalTok{  phi[phi}\OperatorTok{==}\DecValTok{1}\NormalTok{] <-}\StringTok{ }\FloatTok{0.9999} \CommentTok{# avoid NaN of log function}
\NormalTok{  phi[phi}\OperatorTok{==}\DecValTok{0}\NormalTok{] <-}\StringTok{ }\FloatTok{0.0001}
\NormalTok{  f <-}\StringTok{ }\KeywordTok{sum}\NormalTok{(y.}\OperatorTok{*}\KeywordTok{log}\NormalTok{(phi))}\OperatorTok{+}\KeywordTok{sum}\NormalTok{((}\DecValTok{1}\OperatorTok{-}\NormalTok{y.)}\OperatorTok{*}\KeywordTok{log}\NormalTok{(}\DecValTok{1}\OperatorTok{-}\NormalTok{phi))}
\NormalTok{  f <-}\StringTok{ }\OperatorTok{-}\NormalTok{f}
  \KeywordTok{return}\NormalTok{(f)}
\NormalTok{\}}
\end{Highlighting}
\end{Shaded}

\subsubsection{Optimizer using steepest
ascent}\label{optimizer-using-steepest-ascent}

By input a inial guess of parameter, relative stopping criteria
(percentage change in function), and function, using gradient decent
method featuring backtracking line search, you can get the parameter
that minimize the funtion. Note: First arguement of fun. must be the
parameter you want to get, X and y must be set into the default value.

\begin{Shaded}
\begin{Highlighting}[]
\NormalTok{graddec <-}\StringTok{ }\ControlFlowTok{function}\NormalTok{(b,stop,fun)\{}
\NormalTok{  d <-}\StringTok{ }\FloatTok{0.001} \CommentTok{# delta in the finite difference method}
\NormalTok{  alpha <-}\StringTok{ }\FloatTok{0.1} \CommentTok{# any initial alpha}
\NormalTok{  delta <-}\StringTok{ }\OtherTok{Inf}
  \ControlFlowTok{while}\NormalTok{ (delta }\OperatorTok{>}\StringTok{ }\NormalTok{stop)\{}
    \CommentTok{# Calculate gradient for each regressor by Finite Difference Method}
\NormalTok{    b. <-}\StringTok{ }\KeywordTok{matrix}\NormalTok{(b,}\KeywordTok{length}\NormalTok{(b),}\DecValTok{4}\NormalTok{)}
\NormalTok{    g <-}\StringTok{ }\NormalTok{(}\KeywordTok{apply}\NormalTok{(b. }\OperatorTok{+}\StringTok{ }\KeywordTok{diag}\NormalTok{(d,}\DecValTok{4}\NormalTok{),}\DecValTok{2}\NormalTok{,fun)}\OperatorTok{-}\KeywordTok{apply}\NormalTok{(b.,}\DecValTok{2}\NormalTok{,fun))}\OperatorTok{/}\NormalTok{d}
    \CommentTok{# Backtracking (Armijo–Goldstein condition tests to make sure the size of alpha)}
    \ControlFlowTok{if}\NormalTok{(}\KeywordTok{fun}\NormalTok{(b }\OperatorTok{-}\StringTok{ }\NormalTok{alpha}\OperatorTok{*}\NormalTok{g)}\OperatorTok{>}\KeywordTok{fun}\NormalTok{(b)}\OperatorTok{-}\FloatTok{0.1}\OperatorTok{*}\NormalTok{alpha}\OperatorTok{*}\KeywordTok{t}\NormalTok{(g)}\OperatorTok\NormalTok{g)\{}
\NormalTok{      alpha <-}\StringTok{ }\FloatTok{0.5}\OperatorTok{*}\NormalTok{alpha}
      \ControlFlowTok{next}
\NormalTok{    \}}
    \CommentTok{# Make step forward (t+1) and saved in bn}
\NormalTok{    bn <-}\StringTok{ }\NormalTok{b }\OperatorTok{-}\StringTok{ }\NormalTok{alpha}\OperatorTok{*}\NormalTok{g }
    \CommentTok{# Stopping Criteria }
\NormalTok{    delta <-}\StringTok{ }\NormalTok{(}\KeywordTok{abs}\NormalTok{(}\KeywordTok{fun}\NormalTok{(bn)}\OperatorTok{-}\KeywordTok{fun}\NormalTok{(b))}\OperatorTok{/}\KeywordTok{abs}\NormalTok{(}\KeywordTok{fun}\NormalTok{(b)))}
\NormalTok{    b <-}\StringTok{ }\NormalTok{bn}
\NormalTok{  \}}
  \KeywordTok{return}\NormalTok{(b)}
\NormalTok{\}}
\NormalTok{result.gd <-}\StringTok{ }\KeywordTok{graddec}\NormalTok{(}\KeywordTok{c}\NormalTok{(}\DecValTok{0}\NormalTok{,}\DecValTok{0}\NormalTok{,}\DecValTok{0}\NormalTok{,}\DecValTok{0}\NormalTok{),}\FloatTok{1e-5}\NormalTok{,probit.llike)}
\NormalTok{result.gd}
\end{Highlighting}
\end{Shaded}

\begin{verbatim}
## [1]  2.2900259  1.3625441 -0.8485096  0.0949513
\end{verbatim}

\subsubsection{Except the coefficient on constant, others are close, but
still have a big
difference.}\label{except-the-coefficient-on-constant-others-are-close-but-still-have-a-big-difference.}

\subsection{Exercise 4 ----}\label{exercise-4--}

\subsubsection{Optimize Probit}\label{optimize-probit}

\begin{Shaded}
\begin{Highlighting}[]
\NormalTok{b.p <-}\StringTok{ }\KeywordTok{c}\NormalTok{(}\DecValTok{0}\NormalTok{,}\DecValTok{0}\NormalTok{,}\DecValTok{0}\NormalTok{,}\DecValTok{0}\NormalTok{)}
\NormalTok{result.p <-}\StringTok{ }\KeywordTok{optim}\NormalTok{(}\DataTypeTok{par =}\NormalTok{ b.p, probit.llike)}
\NormalTok{result.p}\OperatorTok{$}\NormalTok{par}
\end{Highlighting}
\end{Shaded}

\begin{verbatim}
## [1]  2.81623688  1.23884124 -0.89197772  0.04781058
\end{verbatim}

\subsubsection{Optimizing Logit}\label{optimizing-logit}

\begin{Shaded}
\begin{Highlighting}[]
\NormalTok{logit.llike <-}\StringTok{ }\ControlFlowTok{function}\NormalTok{(b., }\DataTypeTok{y. =}\NormalTok{ ydum,}\DataTypeTok{X. =}\NormalTok{ X)\{}
\NormalTok{  gamma <-}\StringTok{ }\KeywordTok{exp}\NormalTok{(X.}\OperatorTok\NormalTok{b.)}\OperatorTok{/}\NormalTok{(}\DecValTok{1}\OperatorTok{+}\KeywordTok{exp}\NormalTok{(X.}\OperatorTok\NormalTok{b.))}
\NormalTok{  f <-}\StringTok{ }\KeywordTok{sum}\NormalTok{(y.}\OperatorTok{*}\KeywordTok{log}\NormalTok{(gamma))}\OperatorTok{+}\KeywordTok{sum}\NormalTok{((}\DecValTok{1}\OperatorTok{-}\NormalTok{y.)}\OperatorTok{*}\KeywordTok{log}\NormalTok{(}\DecValTok{1}\OperatorTok{-}\NormalTok{gamma))}
\NormalTok{  f <-}\StringTok{ }\OperatorTok{-}\NormalTok{f}
  \KeywordTok{return}\NormalTok{(f)}
\NormalTok{\}}
\NormalTok{b.l <-}\StringTok{ }\KeywordTok{c}\NormalTok{(}\DecValTok{0}\NormalTok{,}\DecValTok{0}\NormalTok{,}\DecValTok{0}\NormalTok{,}\DecValTok{0}\NormalTok{)}
\NormalTok{result.l <-}\StringTok{ }\KeywordTok{optim}\NormalTok{(}\DataTypeTok{par =}\NormalTok{ b.l, logit.llike)}
\NormalTok{result.l}\OperatorTok{$}\NormalTok{par}
\end{Highlighting}
\end{Shaded}

\begin{verbatim}
## [1]  5.066553  2.231654 -1.606238  0.086059
\end{verbatim}

\subsubsection{Optimizing Linear
Probability}\label{optimizing-linear-probability}

\begin{Shaded}
\begin{Highlighting}[]
\NormalTok{result.lp <-}\StringTok{ }\KeywordTok{lm}\NormalTok{(ydum}\OperatorTok{~}\NormalTok{X1}\OperatorTok{+}\NormalTok{X2}\OperatorTok{+}\NormalTok{X3)}
\KeywordTok{summary}\NormalTok{(result.lp)}
\end{Highlighting}
\end{Shaded}

\begin{verbatim}
## 
## Call:
## lm(formula = ydum ~ X1 + X2 + X3)
## 
## Residuals:
##      Min       1Q   Median       3Q      Max 
## -0.90570 -0.26599  0.05805  0.24995  2.35722 
## 
## Coefficients:
##               Estimate Std. Error  t value Pr(>|t|)    
## (Intercept)  0.8795230  0.0134596   65.345   <2e-16 ***
## X1           0.1520890  0.0057445   26.476   <2e-16 ***
## X2          -0.1055427  0.0009698 -108.831   <2e-16 ***
## X3           0.0105571  0.0072385    1.458    0.145    
## ---
## Signif. codes:  0 '***' 0.001 '**' 0.01 '*' 0.05 '.' 0.1 ' ' 1
## 
## Residual standard error: 0.3305 on 9996 degrees of freedom
## Multiple R-squared:  0.5571, Adjusted R-squared:  0.557 
## F-statistic:  4191 on 3 and 9996 DF,  p-value: < 2.2e-16
\end{verbatim}

\subsubsection{Check significance with glm
function}\label{check-significance-with-glm-function}

\begin{Shaded}
\begin{Highlighting}[]
\NormalTok{result.p.glm <-}\StringTok{ }\KeywordTok{glm}\NormalTok{(ydum }\OperatorTok{~}\StringTok{ }\NormalTok{X1 }\OperatorTok{+}\StringTok{ }\NormalTok{X2 }\OperatorTok{+}\StringTok{ }\NormalTok{X3, }\DataTypeTok{family =} \KeywordTok{binomial}\NormalTok{(}\DataTypeTok{link =} \StringTok{"probit"}\NormalTok{), }
                \DataTypeTok{data =} \KeywordTok{data.frame}\NormalTok{(mydata))}
\KeywordTok{coef}\NormalTok{(}\KeywordTok{summary}\NormalTok{(result.p.glm))}
\end{Highlighting}
\end{Shaded}

\begin{verbatim}
##                Estimate Std. Error    z value      Pr(>|z|)
## (Intercept)  2.81677032 0.09725690  28.962164 1.972323e-184
## X1           1.23905407 0.04413997  28.071019 2.213032e-173
## X2          -0.89214080 0.01803881 -49.456742  0.000000e+00
## X3           0.04803623 0.04686155   1.025067  3.053315e-01
\end{verbatim}

\begin{Shaded}
\begin{Highlighting}[]
\NormalTok{result.l.glm <-}\StringTok{ }\KeywordTok{glm}\NormalTok{(ydum }\OperatorTok{~}\StringTok{ }\NormalTok{X1 }\OperatorTok{+}\StringTok{ }\NormalTok{X2 }\OperatorTok{+}\StringTok{ }\NormalTok{X3, }\DataTypeTok{family =} \KeywordTok{binomial}\NormalTok{(}\DataTypeTok{link =} \StringTok{"logit"}\NormalTok{), }
                \DataTypeTok{data =} \KeywordTok{data.frame}\NormalTok{(mydata))}
\KeywordTok{coef}\NormalTok{(}\KeywordTok{summary}\NormalTok{(result.l.glm))}
\end{Highlighting}
\end{Shaded}

\begin{verbatim}
##                Estimate Std. Error    z value      Pr(>|z|)
## (Intercept)  5.06601765 0.18221378  27.802605 4.034324e-170
## X1           2.23093874 0.08251564  27.036555 5.496980e-161
## X2          -1.60595037 0.03611664 -44.465664  0.000000e+00
## X3           0.08672068 0.08425283   1.029291  3.033429e-01
\end{verbatim}

The coefficient varies largely across these three methods, but the sign
on the coefficients are the same, and all X3 coefs are not
significant.For Probit and logit, without calcuating the marginal
effect, we can only interprete the sign. Higher X2 can decrease the
probability of ydum = 1, which means higher X2 is likely to generate Y
below mean. Higher X1 and X3 can increase the probability of ydum = 1,
which means that higher X1 and X3 is likely to genrate Y above mean.
This mathch the sign in the data generating process. For linear
probability model, X1: one unit increase in X1 likely to increase the
likelihood of ydum = 1 by 14\%; X2: one unit increase in X2 likely to
decrease the likelihood of ydum = 1 by 10\%; X3: one unit increase in X3
likely to increase the likelihood of ydum = 1 by 1\%

\subsection{Exercise 5 ---}\label{exercise-5}

\subsubsection{Compute Average Marginal Effect (AME) of X of
probit}\label{compute-average-marginal-effect-ame-of-x-of-probit}

\begin{Shaded}
\begin{Highlighting}[]
\NormalTok{AME.p <-}\StringTok{ }\ControlFlowTok{function}\NormalTok{(result) }\KeywordTok{mean}\NormalTok{(}\KeywordTok{dnorm}\NormalTok{(X}\OperatorTok\NormalTok{result))}\OperatorTok{*}\NormalTok{result }\CommentTok{# result are the parameters}
\KeywordTok{AME.p}\NormalTok{(}\KeywordTok{coef}\NormalTok{(result.p.glm))}
\end{Highlighting}
\end{Shaded}

\begin{verbatim}
##  (Intercept)           X1           X2           X3 
##  0.342005585  0.150443012 -0.108321625  0.005832446
\end{verbatim}

\subsubsection{Compute Average Margeinal Effect of X of
Logit}\label{compute-average-margeinal-effect-of-x-of-logit}

\begin{Shaded}
\begin{Highlighting}[]
\NormalTok{AME.l <-}\StringTok{ }\ControlFlowTok{function}\NormalTok{(result) }\KeywordTok{mean}\NormalTok{(}\KeywordTok{dlogis}\NormalTok{(X}\OperatorTok\NormalTok{result))}\OperatorTok{*}\NormalTok{result}
\KeywordTok{AME.l}\NormalTok{(}\KeywordTok{coef}\NormalTok{(result.l.glm))}
\end{Highlighting}
\end{Shaded}

\begin{verbatim}
##  (Intercept)           X1           X2           X3 
##  0.340761837  0.150062404 -0.108023034  0.005833201
\end{verbatim}

Comment: Except the constant term, both models' average marginal effects
are close to linear probability \#\#\# Compute the Standard Error by
delta method (Probit) Using num derivative function to simplfy the
calualtion

\begin{Shaded}
\begin{Highlighting}[]
\CommentTok{# install.packages("numDeriv")}
\KeywordTok{library}\NormalTok{(}\StringTok{"numDeriv"}\NormalTok{)}
\end{Highlighting}
\end{Shaded}

\begin{verbatim}
## Warning: package 'numDeriv' was built under R version 3.5.2
\end{verbatim}

\begin{Shaded}
\begin{Highlighting}[]
\NormalTok{gradient <-}\StringTok{ }\KeywordTok{jacobian}\NormalTok{(AME.p, }\KeywordTok{coef}\NormalTok{(result.p.glm))}
\NormalTok{cov_matrixv <-}\StringTok{ }\KeywordTok{vcov}\NormalTok{(result.p.glm)}
\NormalTok{se.p <-}\StringTok{ }\KeywordTok{sqrt}\NormalTok{(}\KeywordTok{diag}\NormalTok{(gradient}\OperatorTok\NormalTok{cov_matrixv}\OperatorTok\KeywordTok{t}\NormalTok{(gradient)))}
\NormalTok{se.p}
\end{Highlighting}
\end{Shaded}

\begin{verbatim}
## [1] 0.0096251550 0.0044537875 0.0003841457 0.0056888459
\end{verbatim}

\subsubsection{Compute the Standard Error by delta method
(Logit)}\label{compute-the-standard-error-by-delta-method-logit}

\begin{Shaded}
\begin{Highlighting}[]
\NormalTok{gradient <-}\StringTok{ }\KeywordTok{jacobian}\NormalTok{(AME.l, }\KeywordTok{coef}\NormalTok{(result.l.glm))}
\NormalTok{cov_matrixv <-}\StringTok{ }\KeywordTok{vcov}\NormalTok{(result.l.glm)}
\NormalTok{se.l <-}\StringTok{ }\KeywordTok{sqrt}\NormalTok{(}\KeywordTok{diag}\NormalTok{(gradient}\OperatorTok\NormalTok{cov_matrixv}\OperatorTok\KeywordTok{t}\NormalTok{(gradient)))}
\NormalTok{se.l}
\end{Highlighting}
\end{Shaded}

\begin{verbatim}
## [1] 0.0095907416 0.0044580591 0.0003751031 0.0056660253
\end{verbatim}

Compute the SE of AME by Bootstrap (Probit and Logit)

\begin{Shaded}
\begin{Highlighting}[]
\NormalTok{bootstrapse2 <-}\StringTok{ }\ControlFlowTok{function}\NormalTok{(n,l)\{}
  \CommentTok{# input bootstrap times and type of link function}
\NormalTok{  boot.me <-}\StringTok{ }\KeywordTok{data.frame}\NormalTok{(}\KeywordTok{matrix}\NormalTok{(}\DataTypeTok{nrow=}\DecValTok{4}\NormalTok{, }\DataTypeTok{ncol=}\DecValTok{1}\NormalTok{))[}\OperatorTok{-}\DecValTok{1}\NormalTok{]}
  \ControlFlowTok{for}\NormalTok{ (i }\ControlFlowTok{in} \DecValTok{1}\OperatorTok{:}\NormalTok{n) \{}
    \CommentTok{# sample from existing data to get X.s, y.s}
\NormalTok{    df.s <-}\StringTok{ }\NormalTok{mydata[}\KeywordTok{sample}\NormalTok{(}\KeywordTok{nrow}\NormalTok{(mydata),}\DataTypeTok{size =} \KeywordTok{nrow}\NormalTok{(mydata),}\DataTypeTok{replace =}\NormalTok{ T),]}
\NormalTok{    result <-}\StringTok{ }\KeywordTok{glm}\NormalTok{(ydum }\OperatorTok{~}\StringTok{ }\NormalTok{X1 }\OperatorTok{+}\StringTok{ }\NormalTok{X2 }\OperatorTok{+}\StringTok{ }\NormalTok{X3, }\DataTypeTok{family=}\KeywordTok{binomial}\NormalTok{(}\DataTypeTok{link =}\NormalTok{ l),}\KeywordTok{data.frame}\NormalTok{(df.s))}
    \CommentTok{# calculate ols coefficient in each sample}
\NormalTok{    me <-}\StringTok{ }\KeywordTok{mean}\NormalTok{(}\KeywordTok{dnorm}\NormalTok{(}\KeywordTok{predict.glm}\NormalTok{(result, }\DataTypeTok{type =} \StringTok{"link"}\NormalTok{)))}\OperatorTok{*}\KeywordTok{coef}\NormalTok{(result)}
\NormalTok{    boot.me <-}\StringTok{ }\KeywordTok{cbind}\NormalTok{(boot.me,me)}
\NormalTok{  \}}
  \CommentTok{# Report the SE over all the obtained coefficients}
  \KeywordTok{return}\NormalTok{(}\KeywordTok{data.frame}\NormalTok{(}\DataTypeTok{se =} \KeywordTok{apply}\NormalTok{(boot.me,}\DecValTok{1}\NormalTok{,sd)))}
\NormalTok{\}}
\KeywordTok{bootstrapse2}\NormalTok{(}\DecValTok{49}\NormalTok{,}\StringTok{"probit"}\NormalTok{)}
\end{Highlighting}
\end{Shaded}

\begin{verbatim}
##                      se
## (Intercept) 0.010730568
## X1          0.004796591
## X2          0.001396285
## X3          0.005351611
\end{verbatim}

\begin{Shaded}
\begin{Highlighting}[]
\KeywordTok{bootstrapse2}\NormalTok{(}\DecValTok{49}\NormalTok{,}\StringTok{"logit"}\NormalTok{)}
\end{Highlighting}
\end{Shaded}

\begin{verbatim}
##                      se
## (Intercept) 0.012499986
## X1          0.005756487
## X2          0.001994418
## X3          0.005185732
\end{verbatim}


\end{document}
